% This is a LaTeX document. To output this to PDF, issue "make" in the current
% directory. It will compile on central.
%
% Q. "Oh no! This is terrifying! How do I get this working?!"
% A. $ ssh you2@central.aber.ac.uk  (use PuTTY if you can't)
%    $ git clone git://github.com/cs21120/meta
%    $ cd meta
%    $ cd plan
%    $ make
%    Visit http://users.aber.ac.uk/you2/plan.pdf

\documentclass[12pt,a4paper,titlepage,onecolumn,draft]{report}
% remove draft before submission!
% packages go here
\usepackage{hyperref}

\title{CS22120: Group Project \\ \emph{DRAFT}}
\author{
 Dillon Cuffe \and
  Christopher Edwards \and
  Douglas Gardner \and
  Luke Horwood \and
  Jostein Kristiansen \and
  Ashley Smith \and
  Ben Rainbow \and
  James Woodside
}

\begin{document}
\maketitle
\tableofcontents
% above is the preamble; it is boilerplate and you don't really need to
% understand it. Below is the interesting bits:

\part{Plan} %%%

\section{Introduction}
This document specifies the plan the Group shall use when creating its Group
Project.

\subsection{Purpose of this Document}
The purpose of this document is to describe and specify the Group Plan that will
be used by the second CS21120 Group Project group. Herein, an outline of the
project and its constituent parts are described, in such a way that the Project
can be created with cohesion.

\subsection{Scope}
This document specifies the platform, the architecture, and the target users of
the project. It also describes use-cases and risk analysis of the project; as
well as an overview of deadlines and user-interface design.

\subsection{Objectives}
The main objective of this document is to ensure that all members of the Group
Project are aware of the requirements and objectives of the Project, as well as
aiding the further development of the Project by ensuring that the visual
appearance, functional behaviour, and use-case scenarios of the Project product
is clearly defined and accessible to all.


\section{Overview}

\subsection{Platform Choice}
It has been decided that the platform we will develop the application for is
Android. Android is a free operating system for mobile phones \cite{openhandset},
in which the applications are written in Java \cite{androidapi}. Because the
Android operating system is free, there are no fees payable in order to create
applications (unlike iOS applications, that require a \$99 annual
fee \cite{appledev}.)

\subsection{High-level Architecture}
% TODO

\subsection{Target users}
The target users of our application will be second-year Computer Science
students at Aberystwyth University. The target demographic is familiar with
using computers, smartphones and other technology in their everyday life.
\subsection{Use-cases}

\subsection{User interface design}

\subsection{Gantt chart}

\subsection{Risk analysis}

\begin{thebibliography}{5} % number should be approx. the total number of refs
  \bibitem{openhandset}
    "Android Overview". \emph{Open Handset Alliance.} Web. 30 Oct. 2013.
  \bibitem{androidapi}
    "Application Fundamentals." \emph{Android Developers.} Web. Accessed 30 Oct.
    2013. Accessible at
    \url{http://developer.android.com/guide/components/fundamentals.html}.
  \bibitem{appledev}
    "Choosing an iOS Developer Program". \emph{Apple, Inc.} Web. Accessed 30
    Oct. 2013. Accessible at
    \url{https://developer.apple.com/programs/start/ios/}.
\end{thebibliography}

\end{document}


